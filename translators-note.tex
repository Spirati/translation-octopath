\section*{Alternative Format}
I understand reading in a book format is not for everyone, particularly on mobile devices. While this document is the way I believe this work is best presented, if you find it too cumbersome, hard-to-read, or otherwise inconvenient, you can also read each chapter \href{https://lynne.bearblog.dev/blog/?q=octo84}{online}.

\section*{Acknowledgements}
I'd like to extend my gratitude to everyone in the OTS Discord server who provided consistent feedback and support during the translation process, as well as everyone who shared their appreciation on Tumblr as chapters were being serialized. It's a scary thing releasing a translation into the wild, not knowing if it'll reach anyone at all! Most of all, though, I owe a deep debt of gratitude to 鰤/牙, the original author of this collection. They can be found on \href{https://twitter.com/kiva_blitz}{Twitter}.

\section*{Translator's Note (and a Note on Translator's Notes)}
As a rule, I'm opposed to translator's notes. I think they break the flow of a text, overly insert the presence of the translator between the reader and the original text, and generally reflect a failure on a translator's part to appropriately transform the text in a way that is both faithful and understandable to the target language's audience. That said, when I find that translator's notes are absolutely necessary to explain something that's otherwise untranslatable (or that would be awkward to translate with total fidelity, but which bears preserving), they'll be found as hyperlink-enabled endnotes in their own section at the end of the chapter.
If you have any questions about the original text or my writing choices, I'm more than happy to field them on \href{https://twitter.com/plvpwaa}{Twitter}, \href{https://plvpwaa.tumblr.com}{Tumblr}, or via \href{mailto:plvpwaa@lynnux.org}{email}.

You can read more of my translation work on \href{https://lynne.bearblog.dev}{my blog}.
\\

Happy reading!